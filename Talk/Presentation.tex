\documentclass{beamer}

\mode<presentation>
{
  \usetheme{CambridgeUS}
  \setbeamercovered{transparent}
}

\usepackage[english]{babel}
\usepackage[latin1]{inputenc}
\usepackage{times}
\usepackage[T1]{fontenc} 
% Or whatever. Note that the encoding and the font should match. If T1
% does not look nice, try deleting the line with the fontenc.
\usepackage{amsmath}
\usepackage{algorithmic}

\newcommand{\linespace}{\vskip 0.25cm}

\definecolor{MyForestGreen}{rgb}{0,0.7,0} 
\newcommand{\tableemph}[1]{{#1}}
\newcommand{\tablewin}[1]{\tableemph{#1}}
\newcommand{\tablemid}[1]{\tableemph{#1}}
\newcommand{\tablelose}[1]{\tableemph{#1}}

\definecolor{MyLightGray}{rgb}{0.6,0.6,0.6}
\newcommand{\tabletie}[1]{\color{MyLightGray} {#1}}

% The text in square brackets is the short version of your title and will be used in the
% header/footer depending on your theme.
\title[Intrusion Detection]{Intrusion Detection with \\ Genetic Algorithms and Fuzzy Logic}

% Sub-titles are optional - uncomment and edit the next line if you want one.
% \subtitle{Why does sub-tree crossover work?} 

% The text in square brackets is the short version of your name(s) and will be used in the
% header/footer depending on your theme.
\author[Ireland]{Emma Ireland}

% The text in square brackets is the short version of your institution and will be used in the
% header/footer depending on your theme.
\institute[U of Minn, Morris]
{
  Division of Science and Mathematics \\
  University of Minnesota, Morris \\
  Morris, Minnesota, USA
}

% The text in square brackets is the short version of the date if you need that.
\date[December '13] % (optional)
{December 2013 \\ UMM CSci Senior Seminar Conference}

% Delete this, if you do not want the table of contents to pop up at
% the beginning of each subsection:
\AtBeginSection[]
{
  \begin{frame}<beamer>
    \frametitle{Outline}
    \tableofcontents[currentsection, hideothersubsections]
  \end{frame}
}

\begin{document}

\begin{frame}
  \titlepage
\end{frame}

% For a 20-25 minute senior seminar talk you probably want something like:
% - Two or three major sections (other than the summary).
% - At *most* three subsections per section.
% - Talk about 30s to 2min per frame. So there should probably be between
%   15 and 30 frames, all told.

\section*{Overview}

\subsection*{The Big Picture}

\begin{frame}
  \frametitle{The Big Picture}
  
  \begin{columns}
  \begin{column}{0.6\textwidth}
  \begin{itemize}
  	\item 
	\item 
	\item 
	\item 
	\item 
  \end{itemize}
  \end{column}
  \begin{column}{0.4\textwidth}
   
  \end{column}
  \end{columns}
\end{frame}

\subsection*{Outline}

\begin{frame}
  \frametitle{Outline}
  \tableofcontents[hideallsubsections]
\end{frame}
%%%%%%%%%%%%%%%%%%%%%%%%%%%%%%%%%%%%%%%%%%%%%%%%%%%%%%%%%%%%%%%%%%%%%%%%%%%%%%%%%
%%%%%%%%%%%%%%%%%%%%%%%%%%%%%%%%%%%%%%%%%%%%%%%%%%%%%%%%%%%%%%%%%%%%%%%%%%%%%%%%%
%%%%%%%%%%%%%%%%%%%%%%%%%%%%%%%%%%%%%%%%%%%%%%%%%%%%%%%%%%%%%%%%%%%%%%%%%%%%%%%%%
%%%%%%%%%%%%%%%%%%%%%%%%%%%%%%%%%%%%%%%%%%%%%%%%%%%%%%%%%%%%%%%%%%%%%%%%%%%%%%%%%
\section[Background]{Background}
\subsection{Types of Networking Attacks}
\begin{frame}
  \frametitle{Types of Networking Attacks}
Explain DoS, remote to user, user to root, probe
\end{frame}


\subsection{Detection Methodologies}
\begin{frame}
  \frametitle{Detection Methodologies}
Explain signature-based and anomaly-based detection
\end{frame}


\subsection{Data Sets - KDD99 and RLD09}
\begin{frame}
  \frametitle{KDD99}
	\begin{itemize}
		\item Generated by simulating a military network environment in 1999.
		\item Has long been a standard data set for intrusion detection.
		\item Data in the set is classified as normal or attack activity.

		\linespace
		\linespace
		
		\item KDD99 uses 41 features.
		\begin{itemize}
		  	\item \emph{Features} are properties of a \emph{record}, (either an attack or normal activity), that are used to describe the activity.
		\end{itemize}

	\end{itemize}
\end{frame}


\begin{frame}
  \frametitle{Some Features of KDD99}
	\begin{enumerate}
		\item duration: length of the normal or attack activity
in seconds.
		\item src\_bytes: number of bytes sent from source to destination.
		\item num\_failed\_logins: number of failed login attempts.
		\item root\_shell: returns 1 if root shell is obtained, else returns 0.
		\item num\_access\_files: number of operations on access control files.
		\item srv\_count: number of connections to the same service as the current connection in the past two seconds.
		\item serror\_rate: percentage of connections that have ``SYN" errors.
		\item same\_srv\_rate: percentage of connections to the same service.
	\end{enumerate}
\end{frame}



\begin{frame}
  \frametitle{RLD09}
	\begin{itemize}
		\item RLD09 was created because KDD99 is 14 years old.
		\item Data was captured from a university in Bangkok, Thailand.
		\item The data has 10 million data packets.
		\item 17 different types of attacks - divided into denial of service attacks, probe attacks. It also has normal activity.
		\item 12 features, which include the number of packets, source ports, and destination ports.
	\end{itemize}
\end{frame}


\subsection{Rules}
\begin{frame}
  \frametitle{Rules}
	\begin{itemize}
		\item Elements of one set are separated into different sets in order to differentiate between normal connections and attacks.

		\linespace
		\linespace
		
		\item If <\emph{condition}> then <\emph{action}>.
		
		\linespace
		\linespace
				
		\item Specify the details of a packet such as the IP address or port number.
		\item If a packet matches any of the rules in the intrusion detection system, the system will take appropriate action, which may include stopping the connection or logging off the system.
		
	\end{itemize}
\end{frame}


\subsection{Fuzzy Logic}
\begin{frame}
  \frametitle{Fuzzy Logic}

  \begin{itemize}
  	\item Used to detect patterns that have a behavior that is between normal and unusual.
  	\item If <\emph{condition}> then <\emph{consequence}>.
  	\begin{itemize}
  		\item \emph{condition} is a fuzzy variable and \emph{consequence} is a fuzzy set
  	\end{itemize}
  	\item If the number of packets with the same destination address is 20, and $a$=10, $b$=25, $c$=45, then the degree=.6 and the region=B so the number of packets=medium.
  \end{itemize} 
  
  \begin{columns}
  \begin{column}{0.4\textwidth}
      \includegraphics[width=0.95\textwidth]{../triangleFigure.JPG} 
  \end{column}

  \begin{column}{0.6\textwidth}
  \begin{algorithmic}
	\IF{x is between $a$ and $b$}
		\STATE{degree = $(\textrm{x}-a)/(b-a)$}
	\ELSIF{x is between $b$ and $c$}
		\STATE{degree = $(c-\textrm{x})/(c-b)$}
	\ELSE \STATE{degree = 0.0}
	\ENDIF
  \end{algorithmic}
  \end{column}
  \end{columns}
\end{frame}


\subsection{Genetic Algorithms}
\begin{frame}
  \frametitle{Genetic Algorithms}

\end{frame}


\subsection{Determining the Accuracy of an Algorithm}
\begin{frame}
  \frametitle{Determining the Accuracy of an Algorithm}
Explain training and test set, false positive, false negative, true positive, true negative, detection rate.
\end{frame}
%%%%%%%%%%%%%%%%%%%%%%%%%%%%%%%%%%%%%%%%%%%%%%%%%%%%%%%%%%%%%%%%%%%%%%%%%%%%%%%%%
%%%%%%%%%%%%%%%%%%%%%%%%%%%%%%%%%%%%%%%%%%%%%%%%%%%%%%%%%%%%%%%%%%%%%%%%%%%%%%%%%
%%%%%%%%%%%%%%%%%%%%%%%%%%%%%%%%%%%%%%%%%%%%%%%%%%%%%%%%%%%%%%%%%%%%%%%%%%%%%%%%%
%%%%%%%%%%%%%%%%%%%%%%%%%%%%%%%%%%%%%%%%%%%%%%%%%%%%%%%%%%%%%%%%%%%%%%%%%%%%%%%%%
\section[Genetic Algorithm Implementation]{Genetic Algorithm Implementation}
\subsection{Algorithm Overview}
\begin{frame}
  \frametitle{Algorithm Overview}

\end{frame}


\subsection{Experimental Design and Results}
\begin{frame}
  \frametitle{Experimental Design}

\end{frame}


\begin{frame}
  \frametitle{Results}

\end{frame}
%%%%%%%%%%%%%%%%%%%%%%%%%%%%%%%%%%%%%%%%%%%%%%%%%%%%%%%%%%%%%%%%%%%%%%%%%%%%%%%%%
%%%%%%%%%%%%%%%%%%%%%%%%%%%%%%%%%%%%%%%%%%%%%%%%%%%%%%%%%%%%%%%%%%%%%%%%%%%%%%%%%
%%%%%%%%%%%%%%%%%%%%%%%%%%%%%%%%%%%%%%%%%%%%%%%%%%%%%%%%%%%%%%%%%%%%%%%%%%%%%%%%%
%%%%%%%%%%%%%%%%%%%%%%%%%%%%%%%%%%%%%%%%%%%%%%%%%%%%%%%%%%%%%%%%%%%%%%%%%%%%%%%%%
\section[Fuzzy Genetic Algorithm Implementation]{Fuzzy Genetic Algorithm Implementation}

\subsection{Main Points of Research}
\begin{frame}
  \frametitle{Main Points of Research}
	\begin{itemize}
		\item Detecting new or unknown types of attacks in a network.
		\item The intrusion detection system used is able to identify normal network activity as well as attacks using a fuzzy genetic algorithm.
		
		\linespace
		\linespace
		
		\item Ran experiments using only RLD09, and experiments using KDD99 and RLD09 together.
	\end{itemize}
\end{frame}


\subsection{Fuzzy Algorithm}

\begin{frame}
  \frametitle{Measuring the Probability of a Record Being an Attack}

  \begin{columns}
  \begin{column}{0.4\textwidth}
  Trapezoidal shape
  
  \includegraphics[width=0.95\textwidth]{../trapFigure.jpg}
  
  \begin{itemize}
	\item The parameters are the values of a feature.
  \end{itemize}  
  
  \end{column}


  \begin{column}{0.6\textwidth}
%  \begin{itemize}
%  	\item Fuzzy algorithm
%  \end{itemize}
  
\begin{algorithmic}
\IF{data value is between $b$ and $c$}
  \STATE{prob = 1.0}
\ELSIF{data value is between $a$ and $b$}
  \STATE{
    prob = $(\textrm{data}-a)/(b-a)$
  }
\ELSIF{data value is between $c$ and $d$}
  \STATE{
    prob = $(d-\textrm{data})/(d-c)$
  }
\ELSE \STATE{prob = 0.0}
\ENDIF
\end{algorithmic}

  \end{column}
  \end{columns}
\end{frame}


\begin{frame}
	\frametitle{Encoding of Features and Rules}
	\begin{itemize}
	\item The four parameters are encoded into blocks.
	\item Each block is a feature with values between 0.0 and 7.0.
	\end{itemize}
	
\begin{figure}
\begin{tabular}{|cccc|} \hline
010 & 011 & 100 & 101\\
a=2 & b=3 & c=4 & d=5\\
\hline\end{tabular}
\end{figure}

	\begin{itemize}
	\item A rule has 12 blocks of features, at the end is the type of attack.
	\end{itemize}

\begin{figure}
\begin{tabular}{|cccc|c|cccc|c|} \hline
010 & 011 & 100 & 101   & ...... & 010 & 011 & 101 & 111   & DoS\\
a=2 & b=3 & c=4 & d=5   & ...... & a=2 & b=3 & c=5 & d=7   &\\ 
    &     & Block 1&    &        &     & Block 12& &       & Type\\
\hline\end{tabular}
\end{figure}

\end{frame}


\subsection{Algorithm Overview}

\begin{frame}
	\frametitle{Algorithm Overview}
	
\end{frame}


\subsection{Experimental Design and Results}

\begin{frame}
	\frametitle{Experiments Using Only RLD09}
Experiment 1
	
\end{frame}


\begin{frame}
	\frametitle{Experiments Using Only RLD09}
Experiment 2
	
\end{frame}


\begin{frame}
	\frametitle{Experiments Using Both RLD09 and KDD99}
Experiment 1
	
\end{frame}


\begin{frame}
	\frametitle{Experiments Using Both RLD09 and KDD99}
Experiment 2
	
\end{frame}


\begin{frame}
	\frametitle{Experiments Using Both RLD09 and KDD99}
Experiment 3
	
\end{frame}
%%%%%%%%%%%%%%%%%%%%%%%%%%%%%%%%%%%%%%%%%%%%%%%%%%%%%%%%%%%%%%%%%%%%%%%%%%%%%%%%%
%%%%%%%%%%%%%%%%%%%%%%%%%%%%%%%%%%%%%%%%%%%%%%%%%%%%%%%%%%%%%%%%%%%%%%%%%%%%%%%%%
%%%%%%%%%%%%%%%%%%%%%%%%%%%%%%%%%%%%%%%%%%%%%%%%%%%%%%%%%%%%%%%%%%%%%%%%%%%%%%%%%
%%%%%%%%%%%%%%%%%%%%%%%%%%%%%%%%%%%%%%%%%%%%%%%%%%%%%%%%%%%%%%%%%%%%%%%%%%%%%%%%%
\section[Conclusions]{Conclusions}

\begin{frame}
\frametitle{Conclusions}

\begin{itemize}
  \item 
  
  \linespace
  
  \item 

  \linespace
  
  \item
\end{itemize}


\end{frame}

\begin{frame}
	\frametitle{Thanks!}
	
	Thank you for your time and attention!
		
	\linespace
	\linespace
	
	\begin{center}
	{\huge Questions?}
	\end{center}
\end{frame}

\section*{References}

\begin{frame} 
	\frametitle{References} 
	
	\begin{thebibliography}{lskdjf}
	
  	\end{thebibliography}
	
\end{frame} 

\end{document}


