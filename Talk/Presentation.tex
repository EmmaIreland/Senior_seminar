\documentclass{beamer}

\mode<presentation>
{
  \usetheme{CambridgeUS}
  \setbeamercovered{transparent}
}

\usepackage[english]{babel}
\usepackage[latin1]{inputenc}
\usepackage{times}
\usepackage[T1]{fontenc} 
% Or whatever. Note that the encoding and the font should match. If T1
% does not look nice, try deleting the line with the fontenc.
\usepackage{amsmath}
\usepackage{algorithmic}
\usepackage{multirow}

\newcommand{\linespace}{\vskip 0.25cm}

\definecolor{MyForestGreen}{rgb}{0,0.7,0} 
\newcommand{\tableemph}[1]{{#1}}
\newcommand{\tablewin}[1]{\tableemph{#1}}
\newcommand{\tablemid}[1]{\tableemph{#1}}
\newcommand{\tablelose}[1]{\tableemph{#1}}

\definecolor{MyLightGray}{rgb}{0.6,0.6,0.6}
\newcommand{\tabletie}[1]{\color{MyLightGray} {#1}}

% The text in square brackets is the short version of your title and will be used in the
% header/footer depending on your theme.
\title[Intrusion Detection]{Intrusion Detection with \\ Genetic Algorithms and Fuzzy Logic}

% Sub-titles are optional - uncomment and edit the next line if you want one.
% \subtitle{Why does sub-tree crossover work?} 

% The text in square brackets is the short version of your name(s) and will be used in the
% header/footer depending on your theme.
\author[Ireland]{Emma Ireland}

% The text in square brackets is the short version of your institution and will be used in the
% header/footer depending on your theme.
\institute[U of Minn, Morris]
{
  Division of Science and Mathematics \\
  University of Minnesota, Morris \\
  Morris, Minnesota, USA
}

% The text in square brackets is the short version of the date if you need that.
\date[December '13] % (optional)
{December 2013 \\ UMM CSci Senior Seminar Conference}

% Delete this, if you do not want the table of contents to pop up at
% the beginning of each subsection:
\AtBeginSection[]
{
  \begin{frame}<beamer>
    \frametitle{Outline}
    \tableofcontents[currentsection, hideothersubsections]
  \end{frame}
}

\begin{document}

\begin{frame}
  \titlepage
\end{frame}

% For a 20-25 minute senior seminar talk you probably want something like:
% - Two or three major sections (other than the summary).
% - At *most* three subsections per section.
% - Talk about 30s to 2min per frame. So there should probably be between
%   15 and 30 frames, all told.

\section*{Overview}

\subsection*{The Big Picture}

\begin{frame}
  \frametitle{The Big Picture}
  
  \begin{columns}
  \begin{column}{0.6\textwidth}
  \begin{itemize}
  	\item 
	\item 
	\item 
	\item 
	\item 
  \end{itemize}
  \end{column}
  \begin{column}{0.4\textwidth}
   
  \end{column}
  \end{columns}
\end{frame}

\subsection*{Outline}

\begin{frame}
  \frametitle{Outline}
  \tableofcontents[hideallsubsections]
\end{frame}
%%%%%%%%%%%%%%%%%%%%%%%%%%%%%%%%%%%%%%%%%%%%%%%%%%%%%%%%%%%%%%%%%%%%%%%%%%%%%%%%%
%%%%%%%%%%%%%%%%%%%%%%%%%%%%%%%%%%%%%%%%%%%%%%%%%%%%%%%%%%%%%%%%%%%%%%%%%%%%%%%%%
%%%%%%%%%%%%%%%%%%%%%%%%%%%%%%%%%%%%%%%%%%%%%%%%%%%%%%%%%%%%%%%%%%%%%%%%%%%%%%%%%
%%%%%%%%%%%%%%%%%%%%%%%%%%%%%%%%%%%%%%%%%%%%%%%%%%%%%%%%%%%%%%%%%%%%%%%%%%%%%%%%%
\section[Background]{Background}
\subsection{Types of Networking Attacks}
\begin{frame}
  \frametitle{Types of Networking Attacks}
  \begin{itemize}
  	\item Denial of Service (DoS): attacker makes a machine inaccessible to a user by making it too busy to serve legitimate requests.

  	\linespace
  	\linespace

  	\item Remote to User (R2L): attacker tries to gain access to things a local user would have on the machine.

  	\linespace
  	\linespace
  	
  	\item User to Root (U2R): attacker starts out with access on the machine and then tries to gain root access to the system.

  	\linespace
  	\linespace
  	
  	\item Probe: attacker examines a machine in order to collect information about weaknesses or vulnerabilities that in the future could be used to compromise the system.
  \end{itemize}
\end{frame}


\subsection{Detection Methodologies}
\begin{frame}
  \frametitle{Detection Methodologies}
  \begin{itemize}
  	\item Signature-based detection: compares well-known patterns of attacks that are already in the intrusion detection system against captured events in order to identify a possible attack.
  	\item Anomaly-based detection: looks for patterns of activity that are rare and uncommon.
  \end{itemize}
\end{frame}


\subsection{Data Sets - KDD99 and RLD09}
\begin{frame}
  \frametitle{KDD99}
	\begin{itemize}
		\item Generated by simulating a military network environment in 1999.
		\item Has long been a standard data set for intrusion detection.
		\item Data was processed into 5 million records.
			\begin{itemize}
				\item A record is a sequence of TCP packets, between which data flows to and from a source IP address to a target IP address.
			\end{itemize}
		\item Data in the set is classified as normal or attack activity.
		\item Uses 41 features, which are properties of a record that are used to describe the activity and help to distinguish normal connections from attacks.

	\end{itemize}
\end{frame}


\begin{frame}
  \frametitle{Some Features of KDD99}
	\begin{itemize}
		\item duration: length of the normal or attack activity
in seconds.
		%\item src\_bytes: number of bytes sent from source to destination.
		\item num\_failed\_logins: number of failed login attempts.
		\item root\_shell: returns 1 if root shell is obtained, else returns 0.
		%\item num\_access\_files: number of operations on access control files.
		%\item srv\_count: number of connections to the same service as the current connection in the past two seconds.
		%\item serror\_rate: percentage of connections that have ``SYN" errors.
		%\item same\_srv\_rate: percentage of connections to the same service.
	\end{itemize}
\end{frame}



\begin{frame}
  \frametitle{RLD09}
	\begin{itemize}
		\item RLD09 was created because KDD99 is 14 years old.
		\item Data was captured from a university in Bangkok, Thailand.
		\item 17 different types of attacks (divided into denial of service and probe attacks), and 12 features.
	\end{itemize}
\end{frame}


\subsection{Rules}
\begin{frame}
  \frametitle{Rules}
	\begin{itemize}
		\item A commonly used approach for detecting intrusions and to differentiate between normal connections and attacks is to use rules.
		\item If-Then format: If (\emph{condition}) then (\emph{consequence}).
		\begin{itemize}
			\item The condition is composed of one or more features, and the consequence says if it is an intrusion or not.
			\item If \emph{duration = 4} then \emph{intrusion}.
		\end{itemize}				
	\end{itemize}
\end{frame}


\subsection{Genetic Algorithms}
\begin{frame}
  \frametitle{Genetic Algorithms}
	\begin{itemize}
		\item Search technique used to find solutions to problems.
		\item Mutation: random bits in an individual, or possible solution, are changed.
		\item Selection: individuals that have a better fitness are chosen over the other individuals.
		\item Fitness function: determines the quality of a particular individual.
		\item Crossover: Two individuals swap one of their characteristics with the other to form two new individuals.
	\end{itemize}
\end{frame}


\subsection{Determining the Accuracy of an Algorithm}
\begin{frame}
  \frametitle{Determining the Accuracy of an Algorithm}
%	\begin{itemize}
%		\item False positive (FP): intrusion detection system incorrectly identifies normal activity as being an attack. 
%		\item False negative (FN): intrusion detection system fails to identify harmful activity. 
%		\item True positive (TP): intrusion detection system correctly identifies activities to be attacks. 
%		\item True negative (TN): intrusion detection system correctly
%identifies activities to be normal.
%		\item Detection rate (DR): number of intrusions detected by the system divided
%by the total number of intrusions that happen.
%	\end{itemize}
\begin{table}
\begin{tabular}{cccc}
& & \multicolumn{2}{ c }{Predicted} \\
& & Not Attack & Attack \\ \cline{3-4}
\multicolumn{1}{ c }{\multirow{2}{*}{Actual} } &
\multicolumn{1}{ c| }{Not Attack} & True Negative (TN) & False Positive (FP)\\
\multicolumn{1}{ c }{} &
\multicolumn{1}{ c| }{Attack} & False Negative (FN) & True Positive (TP)\\
\end{tabular}
\end{table}
\linespace
	\begin{itemize}
		\item Detection rate (DR): number of intrusions detected by the system divided by the total number of intrusions that happen.
	\end{itemize}
\end{frame}
%%%%%%%%%%%%%%%%%%%%%%%%%%%%%%%%%%%%%%%%%%%%%%%%%%%%%%%%%%%%%%%%%%%%%%%%%%%%%%%%%
%%%%%%%%%%%%%%%%%%%%%%%%%%%%%%%%%%%%%%%%%%%%%%%%%%%%%%%%%%%%%%%%%%%%%%%%%%%%%%%%%
%%%%%%%%%%%%%%%%%%%%%%%%%%%%%%%%%%%%%%%%%%%%%%%%%%%%%%%%%%%%%%%%%%%%%%%%%%%%%%%%%
%%%%%%%%%%%%%%%%%%%%%%%%%%%%%%%%%%%%%%%%%%%%%%%%%%%%%%%%%%%%%%%%%%%%%%%%%%%%%%%%%
\section[Using Fuzzy Genetic Algorithms]{Using Fuzzy Genetic Algorithms}

\subsection{Fuzzy Algorithm}

\begin{frame}
	\frametitle{Fuzzy Logic}
	\begin{itemize}
		\item Attacks on systems do not always have a fixed pattern, so fuzzy logic is used to detect patterns that have a behavior that is between normal and unusual.
		\item Fuzzy logic rules are similar to the rules described before, except that
consequence is a certainty factor. 
		\begin{itemize}
			\item If (\emph{duration} = 6) then (\emph{probability of it being an attack is 50\%}).
		\end{itemize}
	\end{itemize}
\end{frame}


\begin{frame}
  \frametitle{Measuring the Probability of a Record Being an Attack}

  \begin{columns}
  \begin{column}{0.6\textwidth}
  \includegraphics[width=0.95\textwidth]{../TrapezoidFigure.pdf}
  \end{column}

  \begin{column}{0.4\textwidth}
  Example:
  \begin{itemize}
  	\item Feature: duration (length of the activity in seconds).
  	\item a=1, b=3, c=5, d=7
  	\item The length of the activity is 6 seconds (between c and d).
  	\linespace
  	\item prob = $\frac{d-\textrm{data}}{d-c}$ = $\frac{7-6}{7-5}$ = 0.5
  \end{itemize}

  \end{column}
  \end{columns}
\end{frame}


\begin{frame}
	\frametitle{Encoding of Features and Rules}
	\begin{itemize}
	\item The four parameters are encoded into blocks.
	\item Each block is a feature with values between 0.0 and 7.0.
	\end{itemize}
	
\begin{figure}
\begin{tabular}{|cccc|} \hline
010 & 011 & 100 & 101\\
a=2 & b=3 & c=4 & d=5\\
\hline\end{tabular}
\end{figure}

	\begin{itemize}
	\item A rule has 12 blocks of features, at the end is the type of attack.
	\end{itemize}

\begin{figure}
\begin{small}

\begin{tabular}{|cccc|c|cccc|c|} \hline
010 & 011 & 100 & 101   & ...... & 010 & 011 & 101 & 111   & DoS\\
a=2 & b=3 & c=4 & d=5   & ...... & a=2 & b=3 & c=5 & d=7   &\\ 
    &     & Block 1&    &        &     & Block 12& &       & Type\\
\hline\end{tabular}
\end{small}
\end{figure}

\end{frame}


\subsection{Algorithm Overview}

\begin{frame}
	\frametitle{Algorithm Overview}
	\begin{itemize}
		\item The algorithm generates rules, improves rules, then rules are used to classify the data.
		\item One record is passed into a rule.
		\item Each feature in a record is matched to one block of the rule.
		\item The parameters of each block measure the probability of an attack using the trapezoidal fuzzy rule shape.
		\item The probabilities of each block are then compared with a threshold to determine if the record represents an attack or normal behavior.
	\end{itemize}
\end{frame}

\begin{frame}
	\frametitle{Algorithm}
\begin{columns}
\begin{column}{0.6\textwidth}
\begin{small}
\begin{algorithmic}
\FOR{each record}
  \FOR{each rule}
    \FOR{each feature}
      \STATE{prob = fuzzy();}
      \STATE{totalprob = totalprob + prob;}
    \ENDFOR    
    \IF{totalprob > threshold} 
      \STATE{class is attack;}
      \ELSE \STATE{class is normal;}
    \ENDIF
  \ENDFOR
  \STATE{find $A$, $B$, $\alpha$, and $\beta$  %// $A$: \# of attack records. $B$: \# of normal records. $\alpha$: \# of attack records correctly identified as attack. $\beta$: \# of normal records incorrectly classified as attack.}
  
  } 
\ENDFOR
\STATE{calculate fitness}
\STATE{crossover(), mutation()}

\end{algorithmic}
\end{small}
\end{column}

\begin{column}{0.4\textwidth}
	Fitness function:
	\begin{equation*}
	\frac{\alpha}{A} - \frac{\beta}{B}
	\end{equation*}

	$A$: \# of attack records.
	
	$B$: \# of normal records. 

	$\alpha$: \# of attack records correctly identified as attack.

	$\beta$: \# of normal records incorrectly classified as attack.
	
\end{column}
\end{columns}
\end{frame}


\subsection{Experimental Design and Results}

\begin{frame}
	\frametitle{Experiments}
	\begin{itemize}
		\item A variety of experiments were run. Two experiments used just RLD09. Three experiments used both the RLD09 and KDD99 in order to compare how the fuzzy GA would perform on both.
	\end{itemize}
\end{frame}


\begin{frame}
	\frametitle{Experiments Using Only RLD09}
	Experiment 1
	\begin{itemize}
		\item Fuzzy GA was used to create DoS and probe detection rules, then the rules were verified with known attack types.
		\item Two steps in the training process: find a DoS rule, find a probe rule. Both of these rules were then used together in the testing process to identify attacks from the testing data set.
		\item 10,000 records were used for the training set, 26,500 records were used for the test set.
	\end{itemize}
\end{frame}


\begin{frame}
	\frametitle{Experiments Using Only RLD09}
	Experiment 1 Results
\begin{table}
\begin{small}
\begin{tabular}{lllllll}
 & Attack & Normal & Total & FP(\%) & FN(\%) & DR(\%)\\
DoS Training & 1499 & 8501 & 10000 & 1.46 & 47.50 & 91.64\\
Probe Training & 2496 & 7504 & 10000 & 1.83 & 15.38 & 94.79\\
Testing & 10500 & 16000 & 26500 & 1.13 & 4.10 & 97.92\\
\end{tabular}
\end{small}
\end{table}
\end{frame}


\begin{frame}
	\frametitle{Experiments Using Only RLD09}
Experiment 2
	\begin{itemize}
		\item Attacks were pulled out of the training set and kept for unknown data testing. This was to test that the fuzzy GA could detect unknown attacks.
		\item Used fuzzy GA and a decision tree algorithm.
		\item For each test case there were 13 attack types plus normal activity that were in the training data set. Three attack types were used for the unknown testing data set.
	\end{itemize}
\end{frame}


\begin{frame}
	\frametitle{Experiments Using Only RLD09}
	Experiment 2 Results
	
\begin{table}
\begin{footnotesize}
\begin{tabular}{llll}
Test & Unknown & Decision & Fuzzy\\
Case & Attacks & Tree DR (\%) & Genetic DR (\%)\\ \hline

1 & Adv Port Scan (Probe) & Avg = & Avg =\\
  & Ack Scan (Probe)                 & 98.33 & 100\\
  & Xmas Tree (Probe)                 &                 &\\ \hline

2 & UDP Flood (DoS) & Avg = & Avg =\\
  & Host Scan (Probe) & 46.65 & 99.80\\
  & UDP Scan (Probe) & &\\ \hline

3 & Jping (DoS) & Avg = & Avg =\\
  & Syn Scan (Probe) & 99.70 & 98.75\\
  & Fin Scan (Probe) & &\\ \hline

4 & UDP Flood (DoS) & Avg = & Avg =\\
  & RCP Scan (Probe) & 70.35 & 98.15\\
  & Fin Scan (Probe) & &\\ \hline

5 & Http Flood (DoS) & Avg = & Avg =\\
  & RCP Scan (Probe) & 99.94 & 97.50\\
  & Fin Scan (Probe) & &\\
\hline\end{tabular}
\end{footnotesize}
\end{table}
	
\end{frame}


\begin{frame}
	\frametitle{Experiments Using Both RLD09 and KDD99}
Experiment 1
	\begin{itemize}
		\item Used fuzzy GA to classify normal activity and attacks from KDD99 and RLD09.
	\end{itemize}
	
\begin{table}
\begin{tabular}{cccccc}
Data set & Attack & Normal & FP (\%) & FN (\%) & DR (\%)\\ \hline
KDD99 & 160,117 & 39,337 & 0.13 & 1.55 & 98.72\\
RLD09 & 10,500 & 16,000 & 1.14 & 3.39 & 97.97\\
\end{tabular}
\end{table}
\end{frame}


\begin{frame}
	\frametitle{Experiments Using Both RLD09 and KDD99}
Experiment 2
	\begin{itemize}
		\item Used the fuzzy GA to classify types of attacks in KDD99.
		\item 158,597 records of DoS attacks. 1,500 records of probe attacks.
	\end{itemize}
\begin{table}
\begin{tabular}{cccccc}
Test & Attack & Type & FP (\%) & FN (\%) & DR (\%)\\ \hline
1 & Back & DoS & 85.33 & 0.00 & 16.56\\
2 & Smurf & DoS & 0.76 & 0.10 & 99.73\\
3 & Portsweep & Probe & 6.40 & 0.00 & 93.66\\
4 & Satan & Probe & 0.74 & 3.75 & 99.22\\
\end{tabular}
\end{table}

\end{frame}


\begin{frame}
	\frametitle{Experiments Using Both RLD09 and KDD99}
Experiment 3
	\begin{itemize}
		\item Used the fuzzy GA to classify types of attacks in RLD09.
		\item 6,400 records of DoS attacks. 10,400 records of probe attacks.
	\end{itemize}
\begin{table}
\begin{tabular}{cccccc}
Test & Attack & Type & FP (\%) & FN (\%) & DR (\%)\\ \hline
1 & Smurf & DoS & 0.02 & 0 & 99.98\\
2 & UDP Flood & DoS & 11.06 & 0 & 89.59\\
3 & Ackscan & Probe & 0.03 & 0 & 99.97\\
4 & Synscan & Probe & 0.65 & 4.2 & 99.24\\
\end{tabular}
\end{table}

\end{frame}
%%%%%%%%%%%%%%%%%%%%%%%%%%%%%%%%%%%%%%%%%%%%%%%%%%%%%%%%%%%%%%%%%%%%%%%%%%%%%%%%%
%%%%%%%%%%%%%%%%%%%%%%%%%%%%%%%%%%%%%%%%%%%%%%%%%%%%%%%%%%%%%%%%%%%%%%%%%%%%%%%%%
%%%%%%%%%%%%%%%%%%%%%%%%%%%%%%%%%%%%%%%%%%%%%%%%%%%%%%%%%%%%%%%%%%%%%%%%%%%%%%%%%
%%%%%%%%%%%%%%%%%%%%%%%%%%%%%%%%%%%%%%%%%%%%%%%%%%%%%%%%%%%%%%%%%%%%%%%%%%%%%%%%%
\section[Using Genetic Algorithms]{Using Genetic Algorithms}
\subsection{Algorithm Overview}
\begin{frame}
  \frametitle{Algorithm Overview}
	\begin{itemize}
		\item The system is divided into 2 phases: a precalculation phase and a detection phase.
		\item In the precalculation phase, a set of chromosomes are created using training data. It takes network data as an input and outputs a set of chromosomes. Then this set of chromosomes is used in the detection phase.
		\item In the detection phase, selection, crossover, and mutation occur, and then the type of the data (whether it is an attack or normal behavior) is predicted.
	\end{itemize}
\end{frame}


\subsection{Experimental Design and Results}
\begin{frame}
  \frametitle{Experimental Design}
  \begin{columns}

  \begin{column}{0.5\textwidth}
	\begin{itemize}
		\item KDD99 data set.
		\item Used only the numerical features of KDD99 (34 out of 41 total features).
		\item Training set: 494,021 records (396,741 of them are attacks). Test set: 311,029 records (250,436 of them are attacks).
	\end{itemize}
  \end{column}
  
  \begin{column}{0.5\textwidth}
  \begin{table}
\caption{Number of records}
\begin{tabular}{lll}
  & Training & Testing \\
Normal & 97,280 & 60,593\\
Probe & 4,107 & 4,166\\
DoS         & 391,458 & 229,853\\
U2R & 52 & 228\\
R2L & 1,124 & 16,189\\
Total & 494,021 & 311,029\\
\end{tabular}
\end{table}
  \end{column}
  \end{columns}
\end{frame}


\begin{frame}
  \frametitle{Results}
\begin{table}
\begin{small}
\begin{tabular}{cccccccc}
& & \multicolumn{5}{ c }{Predicted} \\
& & Normal & Probe & DoS & U2R & R2L & \% Correct \\ \cline{3-8}
\multicolumn{1}{ c }{\multirow{2}{*}{Actual} } &
\multicolumn{1}{ c| }{Normal} & 42138 & 1421 & 15835 & 486 & 713 & 69.5\\
\multicolumn{1}{ c }{} &
\multicolumn{1}{ c| }{Probe} & 398 & 2963 & 654 & 2 & 149 & 71.1 \\
\multicolumn{1}{ c }{\multirow{2}{*}{} } &
\multicolumn{1}{ c| }{Dos} & 921 & 432 & 228489 & 1 & 10 & 99.4\\
\multicolumn{1}{ c }{} &
\multicolumn{1}{ c| }{U2R} & 146 & 21 & 8 & 43 & 10 & 18.9\\
\multicolumn{1}{ c }{} &
\multicolumn{1}{ c| }{R2L} & 11191 & 578 & 3398 & 141 & 881 & 5.4\\

\multicolumn{1}{ c }{} &
\multicolumn{1}{ c| }{} & & & & & & \\

\multicolumn{1}{ c }{\% Correct} &
\multicolumn{1}{ c| }{} & 76.9 & 54.7 & 92.0 & 6.4 & 50.0 \\
\end{tabular}
\end{small}
\end{table}
\end{frame}
%%%%%%%%%%%%%%%%%%%%%%%%%%%%%%%%%%%%%%%%%%%%%%%%%%%%%%%%%%%%%%%%%%%%%%%%%%%%%%%%%
%%%%%%%%%%%%%%%%%%%%%%%%%%%%%%%%%%%%%%%%%%%%%%%%%%%%%%%%%%%%%%%%%%%%%%%%%%%%%%%%%
%%%%%%%%%%%%%%%%%%%%%%%%%%%%%%%%%%%%%%%%%%%%%%%%%%%%%%%%%%%%%%%%%%%%%%%%%%%%%%%%%
%%%%%%%%%%%%%%%%%%%%%%%%%%%%%%%%%%%%%%%%%%%%%%%%%%%%%%%%%%%%%%%%%%%%%%%%%%%%%%%%%
\section[Conclusions]{Conclusions}

\begin{frame}
\frametitle{Conclusions}

\begin{itemize}
  \item The use of genetic algorithms and fuzzy logic in intrusion detection are effective ways of detecting attacks.
  \item The fuzzy genetic algorithm had a higher detection rate than a decision tree algorithm in most cases.
  \item Fuzzy genetic algorithms are good at detecting unknown attacks.
\end{itemize}


\end{frame}

\begin{frame}
	\frametitle{Thanks!}
	
	Thank you for your time and attention!
		
	\linespace
	\linespace
	
	\begin{center}
	{\huge Questions?}
	\end{center}
\end{frame}

\section*{References}

\begin{frame} 
	\frametitle{References} 
	
	\begin{thebibliography}{lskdjf}
	
  	\end{thebibliography}
	
\end{frame} 

\end{document}


